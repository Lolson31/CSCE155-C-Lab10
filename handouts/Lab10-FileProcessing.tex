\documentclass[12pt]{scrartcl}


\setlength{\parindent}{0pt}
\setlength{\parskip}{.25cm}

\usepackage{graphicx}

\usepackage{xcolor}

\definecolor{darkred}{rgb}{0.5,0,0}
\definecolor{darkgreen}{rgb}{0,0.5,0}
\usepackage{hyperref}
\hypersetup{
  letterpaper,
  colorlinks,
  linkcolor=red,
  citecolor=darkgreen,
  menucolor=darkred,
  urlcolor=blue,
  pdfpagemode=none,
  pdftitle={Introduction To Git},
  pdfauthor={Christopher M. Bourke},
  pdfcreator={$ $Id: cv-us.tex,v 1.28 2009/01/01 00:00:00 cbourke Exp $ $},
  pdfsubject={PhD Thesis},
  pdfkeywords={}
}

\definecolor{MyDarkBlue}{rgb}{0,0.08,0.45}
\definecolor{MyDarkRed}{rgb}{0.45,0.08,0}
\definecolor{MyDarkGreen}{rgb}{0.08,0.45,0.08}

\definecolor{mintedBackground}{rgb}{0.95,0.95,0.95}
\definecolor{mintedInlineBackground}{rgb}{.90,.90,1}

%\usepackage{newfloat}
\usepackage[newfloat=true]{minted}
\setminted{mathescape,
               linenos,
               autogobble,
               frame=none,
               framesep=2mm,
               framerule=0.4pt,
               %label=foo,
               xleftmargin=2em,
               xrightmargin=0em,
               startinline=true,  %PHP only, allow it to omit the PHP Tags *** with this option, variables using dollar sign in comments are treated as latex math
               numbersep=10pt, %gap between line numbers and start of line
               style=default, %syntax highlighting style, default is "default"
               			    %gallery: http://help.farbox.com/pygments.html
			    	    %list available: pygmentize -L styles
               bgcolor=mintedBackground} %prevents breaking across pages
               
\setmintedinline{bgcolor={mintedBackground}}
\setminted[text]{bgcolor={mintedBackground},linenos=false,autogobble,xleftmargin=1em}
%\setminted[php]{bgcolor=mintedBackgroundPHP} %startinline=True}
\SetupFloatingEnvironment{listing}{name=Code Sample}
\SetupFloatingEnvironment{listing}{listname=List of Code Samples}


\title{CSCE 155 - C}
\subtitle{Lab 10 - File I/O}
\author{Dr.\ Chris Bourke}
\date{~}

\begin{document}

\maketitle

\section*{Prior to Lab}

Before attending this lab:
\begin{enumerate}
  \item Read and familiarize yourself with this handout.
  \item Review the following free textbook resources:
	\begin{itemize}
  	  \item \url{http://en.wikibooks.org/wiki/C_Programming/File_IO}
	  \item \url{http://www.cs.cf.ac.uk/Dave/C/node18.html}
	\end{itemize}
\end{enumerate}

\section*{Peer Programming Pair-Up}

To encourage collaboration and a team environment, labs will be
structured in a \emph{pair programming} setup.  At the start of
each lab, you will be randomly paired up with another student 
(conflicts such as absences will be dealt with by the lab instructor).
One of you will be designated the \emph{driver} and the other
the \emph{navigator}.  

The navigator will be responsible for reading the instructions and
telling the driver what to do next.  The driver will be in charge of the
keyboard and workstation.  Both driver and navigator are responsible
for suggesting fixes and solutions together.  Neither the navigator
nor the driver is ``in charge.''  Beyond your immediate pairing, you
are encouraged to help and interact and with other pairs in the lab.

Each week you should alternate: if you were a driver last week, 
be a navigator next, etc.  Resolve any issues (you were both drivers
last week) within your pair.  Ask the lab instructor to resolve issues
only when you cannot come to a consensus.  

Because of the peer programming setup of labs, it is absolutely 
essential that you complete any pre-lab activities and familiarize
yourself with the handouts prior to coming to lab.  Failure to do
so will negatively impact your ability to collaborate and work with 
others which may mean that you will not be able to complete the
lab.  

\section{Lab Objectives \& Topics}
At the end of this lab you should be familiar with the following
\begin{itemize}
  \item Understand the differences between binary and plaintext data
  \item How to read from a file and process information
  \item How to write to a file to persist information
  \item Have some exposure to other topics such as XML and sorting
\end{itemize}

\section{Background}

The life span of most program is short--measured in seconds or 
microseconds.  For data to be useful, it needs to last beyond the 
typical program.  This is known as data persistence.  One mechanism 
for persisting data is to store it in a file.  Files can generally consist 
of raw binary data or plaintext.  In either case, the data needs to be 
structured in some manner for a program to be able to read and write 
it.

In C, file I/O (input/output) is facilitated through several standard 
functions provided by the standard input/output (\mintinline{c}{<stdio.h>})
library, including those found in Table \ref{table:functions}.

\begin{table}
\centering
\begin{tabular}{l|l}
Function & Purpose \\
\hline\hline
\mintinline{c}{fopen()} & Opens a file for reading (r) or writing (w) \\
\mintinline{c}{fclose()} & Closes the file stream passed to the function \\
\mintinline{c}{fprintf}, \mintinline{c}{fscanf} & Output/input from a file stream in a structured manner \\
\mintinline{c}{fgets()}, \mintinline{c}{fputs()} & Inputs/outputs an entire string (line) from/to a file \\
\mintinline{c}{fwrite()}, \mintinline{c}{fread()} & Inputs/outputs binary data \\
\end{tabular}
\caption{Several File I/O Functions}
\label{table:functions}
\end{table}

You will use these functions to complete the activities in this lab.  For 
more details, refer to a standard C reference on how to use these functions.

\subsection*{String Tokenizing}

Often times it is necessary to process formatted data.  One common 
solution is to tokenize a string along some delimiter.  For example, 
given a string of comma-separated-values (CSV) such as 
\mintinline{text}{Sandberg,Ryne,1060 West Addison Street,Chicago,IL,60613}
we may want to process each field, so we would want to split the string using 
the comma as a delimiter into its constituent components (last name, first name, 
address, city, state, zip for example).

The standard string library provides a function for doing this: 

\mintinline{c}{char *strtok(char *str, const char *delim);}

The way to use it is as follows: the first argument is the string to be tokenized, 
the second is the delimiter (or list of delimiters to tokenize on).  The return 
value is the next token with the delimiter removed.  The first time you call 
\mintinline{c}{strtok}, the first argument should be the string to be tokenized.  
Subsequent calls should use pass \mintinline{c}{NULL} to tokenize on the same 
string (otherwise tokenization begins anew on the string).  The same delimiter 
should be passed each time.  The tokenization ends when \mintinline{c}{strtok} 
returns \mintinline{c}{NULL}.  Note: the observant coder will notice that the first 
argument is \emph{not} a \mintinline{c}{const char *}.  As such you can expect 
\mintinline{c}{strtok} to \emph{change} the provided string as it tokenizes, so 
be careful.

\section{Activities}

Clone the code for this lab from GitHub using the following URL: 
\url{https://github.com/cbourke/CSCE155-C-Lab10}.

\subsection{Plaintext versus Binary Data}

In general, data files can contain either binary data (a collection of 0s and 1s) 
or plaintext data (ASCII).  Binary data is generally readable only by a computer 
or program that interprets the 0s and 1s as different types of data (7-bit ASCII 
characters, 32-bit integers, etc.) while ASCII text is readable by humans, but 
may need additional formatting and data conversions to be handled by a 
program.  In this first activity you will get some experience in the contrast 
between these two types of data.

\subsubsection*{Instructions}

In this exercise, you will work with a pre-written program that opens a file 
containing census data on states from the 2010 census.  
\begin{enumerate}
  \item Open the \mintinline{text}{stateData.txt} data file which is in the
  \mintinline{text}{data} folder and observe its contents 
	(do not edit this file)
  \item Examine what the \mintinline{text}{stateData.c} program is doing; 
	compile and run it
  \item The program will create a binary output file, \mintinline{text}{stateData.dat} 
	in the same \mintinline{text}{data} directory.
	\begin{enumerate}
	  \item The unix command \mintinline{text}{file} is a utility that determines the 
		type of a file; examples:
		
		\begin{minted}{text}
		file stateData.txt
		file stateData.dat
		\end{minted}
	  \item Open the \mintinline{text}{stateData.dat} file in your preferred 
	    text editor and observe its contents
	  \item The unix command \mintinline{text}{ls -la} will list some detailed 
		statistics of all the files in a directory.  Run it and observe the sizes of the 
		files (5th column in bytes).
	\end{enumerate}
  \item Answer the questions on your worksheet and move on to the next activity.
\end{enumerate}

\subsection{File Output}

Extensible Markup Language (XML) is a markup language that defines a set 
of rules for formatting data in a file that is both human readable and can be 
processed by a machine.  Each piece of data is semantically marked-up to 
indicate what that data represents.  This enables data to be more portable 
and interoperable across different programs and different programming 
languages.  Many tools and frameworks have been developed around its 
usage.  For example, the state population data may be encoded in XML 
as follows.

\begin{minted}{xml}
<STATES>
  <STATE>
    <NAME>Nebraska</NAME>
    <POPULATION>1826341</POPULATION>
  </STATE>
  ...
  <STATE>
    <NAME>Ohio</NAME>
    <POPULATION>11536504</POPULATION>
  </STATE>
</STATES>
\end{minted}

\subsubsection*{Instructions}

Modify the \mintinline{text}{stateData.c} program by implementing 
(and calling) the function:

\mintinline{c}{void toXMLFile(char **states, int *populations, int numStates);}

\begin{enumerate}
  \item The function should open a file for writing, \mintinline{text}{stateData.xml}
  \item It should create an XML file containing marked up data on all 50 states 
	as in the example above.  
  \item Hint: a convenience function, \mintinline{text}{rtrim} has been provided 
	to you that trims the trailing whitespace from a string and returns a pointer 
	to the new, trimmed string.
  \item Answer the questions on your worksheet and demonstrate your program 
	to a lab instructor.
\end{enumerate}
	
\subsection{File Input \& Data Processing}

Reading data from a file is often done in order to process and aggregate it to 
get additional results.  In this activity you will read in data from a file containing 
win/loss data from the 2011 Major League Baseball season.  Specifically, the 
file \mintinline{text}{data/mlb_nl_2011.txt} contains data about each National League 
team.  Each line contains a team name followed by the number of wins and 
number of losses during the 2011 season.  You will open this file and process 
the information to output a list of teams followed by their win percentage 
(number of wins divided by the total number of games) from highest to lowest.

\subsubsection*{Instructions}

\begin{enumerate}
  \item Open the \mintinline{text}{mlb.c} C source files.  Much of the program 
  	has already been provided for you, including a convenience function to 
	sort the lists of teams and their win percentages as well as a function 
	to output them.
  \item Add code to open the data file and read in the team names, wins and 
	losses and populate the \mintinline{text}{teams[]} and \mintinline{text}{winPercentages[]} 
	arrays with the appropriate data
  \item Call the sort and output functions to sort and display your results
  \item Answer the questions on your worksheet and demonstrate your working 
  	program to a lab instructor
\end{enumerate}

\section{Advanced Activity (Optional)}

\begin{enumerate}
  \item When we sorted the baseball teams and their win percentages, we had 
  	to do all the ``bookkeeping'' ourselves: that is, we had to swap elements 
	in both arrays to make sure that the $i$-th team name matched up with the 
	$i$-th win percentage.  A much better way would have been to define a 
	user defined struct type to hold the team name and win percentage in 
	the same structure.  Redesign the program to use such a struct.
  \item In general, there are no restrictions on the length of a line in a plaintext 
	file (and in a data file, there is not even a concept of a ``line'').  For the best 
	readability, however, it is best to keep lines to a limited, consistent length.  
	One common maximum length for monotype font is 72 characters per 
	line (CPL).  A plaintext file, \mintinline{text}{star_wars.txt} has been provided (the 
	original draft script of the movie Star Wars) that contains some very long 
	lines.  Write a program to read in the file, line by line.  If the line exceeds 
	the 72 CPL limit, break it up into multiple lines (but do not break up individual 
	words).  Output the resulting well-formatted file to a separate file.
\end{enumerate}

\end{document}
